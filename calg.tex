\documentclass[italian]{article}

\usepackage{babel}
\usepackage[utf8]{inputenc}
\usepackage{amsmath}
\usepackage{amssymb}
\usepackage{amsthm}
\usepackage{mathtools}
\usepackage{xfrac}
\usepackage{stmaryrd}
\usepackage{tikz-cd}

\usepackage{hyperref}

\title{Diario di Algebra Commutativa}

\author{G. Susanna}
\date{}

% NOTE: dvi files doesn't support links.
% \href{https://it.wikipedia.org/wiki/Gallus_gallus_domesticus}{Gallina}    


\begin{document}
    \maketitle
    \tableofcontents
    
    \section{Notazione}
    
    In generale $A,B$ indicano anelli commutativi con unità, in generale 
    con la parola anello verrà indicato un anello commutativo con unità.
    Con $A\left[t\right]$ l'anello dei polinomi in una variabile generato 
    dall'anello $A$. Con $A\llbracket t \rrbracket$ l'anello delle serie formali su $A$.
    Con $\operatorname{Spec}(A)$ si indica l'insieme degli ideali primi propri di $A$.
    Con $\operatorname{Specm}(A)$ si indica l'insieme degli ideali massimali di
    $A$.

    \section{Anelli}
    % 
    % Cose dette da ballico a lezione, con corrispettivo link alla lezione.
    %
    \subsection{Lezione 1}
    Le lezioni sono disponibili sottoforma di 
    \href{https://didatticaonline.unitn.it/dol/course/view.php?id=23268}{videolezioni
    su moodle}.
        \begin{enumerate}
          \item[(1a)] Spiegazione del corso, libri.
          \item[(1a-1b)] Definizione di anello commutativo con unità e discussione dei casi banali.
          \item[(1b)] Definizione di omomorfismo e isomorfismo tra anelli.
          \item[(1b)] Alcune proprietà degli omomorfismi $f \colon A \to B$ di anelli:
          	\begin{enumerate}
          		\item $f(0_A) = 0_B$.
          		\item Se $A = \{0\}$ esiste $f$ omomorfismo tra anelli se e solo se $B = \{0\}$.
          		\item Se $B = \{0\}$ allora per ogni anello $A$ esiste un omomorfismo che è quello 
          			costantemente nullo.
          		\item Se e solo se un omomorfismo biettivo, è un isomorfismo.
          	\end{enumerate}
          \item[(1c)] Alcune proprietà degli omomorfismi $f \colon A \to B$ 
            di anelli:
                \begin{enumerate}
                  \item $f(A)$ è un sottoanello di $B$.
                  \item Sia $\ker(f) \coloneqq f^{-1}(\left\{ 0 \right\})$,
                    allora $\ker(f) = \left\{ 0 \right\}$ sse $f$ iniettiva. 
                \end{enumerate}
          \item[(1c)] Definizione di ideale di un anello e esempio sugli interi.
          \item[(1c)] Definizione ideale improprio $I$  e caratterizzazione: improprio sse
            $1 \in I$.
          \item[(1c-1d)] Biezione tra nucleo di funzioni e ideali di un anello. 
        \end{enumerate}

    \subsection{Lezione 2}
    Le lezioni sono disponibili sottoforma di 
    \href{https://didatticaonline.unitn.it/dol/course/view.php?id=23268}{videolezioni
    su moodle}.
    
    \begin{enumerate}
      \item[(2e)] Primo teorema di isomorfismo tra anelli e fattorizzazione degli 
      	omomorfismi.
      \item[(2e)] C'è una biezione tra gli ideali di $\sfrac{A}{I}$ e gli ideali di $A$ che contengono 
		  $I$.
      \item[(2e)] Enunciato del Terzo teorema di isomorfismo, dati $I, J \subseteq A$ ideali allora
      	\begin{equation*}
      		\dfrac{\sfrac{A}{I}}{\sfrac{J}{I}} \simeq \sfrac{A}{J}
      	\end{equation*}
      \item[(2e-2f)] Costruzione dell'anello dei polinomi in una variabile a partire 
      		da un anello.
      \item[(2g)] Costruzione per induzione dell'anello dei polinomi in più variabili.
      \item[(2g)] Costruzione dell'anello delle serie formali a partire da un anello.
      \item[(2g)] Definizione di ideale primo.
      \item[(2g)] Definizione di ideale massimale.
    \end{enumerate}

    \subsection{Lezione 3}
    Le lezioni sono disponibili sottoforma di 
    \href{https://didatticaonline.unitn.it/dol/course/view.php?id=23268}{videolezioni
    su moodle}.
    
    \begin{enumerate}
      \item[(3a)] Riassunto delle precedenti lezioni.
      \item[(3a)] Definizione di alcune operazioni sugli ideali:
      \begin{enumerate}
      	\item Data una famiglia di ideali $\{I_a\}_{a \in \Gamma}$ su un anello $A$ allora 
      	\begin{equation*}
	      	\bigcap_{a \in \Gamma} I_a \subseteq A
      	\end{equation*}
      	è un ideale.
      	\item Dato $A$ anello e $S \subset A$ sottoinsieme qualunque, allora 
      	\begin{equation*}
	      	(S) \coloneqq \bigcap_{s \in \Gamma_S} I_s 
      	\end{equation*} 
        con $\{I_s\}_{s \in \Gamma_S}$ è la famiglia degli ideali di $A$ contenenti $S$. Allora $(S)$ è l'ideale minimale che contiene $S$.
      \end{enumerate}
  	  \item[(3b)] Definizione di ideale principale.
  	  \item[(3b)] Definizione di altre operazioni sugli ideali:
  	  	\begin{enumerate}
  	  		\item Dati due ideali $I,J \subseteq A$ si definisce la somma come 
  	  			\begin{equation*}
  	  				I + J \coloneqq (I \cup J)
  	  			\end{equation*}
    		ovvero il minimo ideale che contiene l'unione dei due ideali.
    		\item In particolare vale la l'equivalenza di definizioni: 
    		\begin{equation*}
    			I + J = \left\{ a + b \,\middle|\, a \in I, b \in J \right\}
    		\end{equation*}
    		\item La somma finita di $n$ ideali è commutativa e associativa.
    		\item Siano $I, J, K \subseteq A$ ideali allora 
    		\begin{equation*}
	    		(I \cap K) + (J \cap K) \subseteq (I + J) \cap K 
    		\end{equation*}
    		\item Non vale in generale l'uguaglianza, si veda l'Esempio 1.2.1\cite{vergura}.
  	  	\end{enumerate}
      \item[(3b)] Dati $I, J\subseteq A$ ideali, allora 
      	\begin{equation*}
	      	IJ \coloneqq (\left\{ab \,\middle|\, a\in I, b \in J\right\})
      	\end{equation*}
      è ancora un ideale ed è detto ideale prodotto.
      \item[(3b)] Alcune proprietà del prodotto di ideali
      \begin{enumerate}
      	\item Dati due ideali $I,J \subseteq A$ allora 
      		\begin{equation*}
	      		IJ \subseteq I \cap J
      		\end{equation*}
      	\item Non vale sempre l'uguaglianza, si prenda $p$ primo in $\mathbb{Z}$, 
      	allora basta prendere $I = J = (p)$.
      \end{enumerate}
  	  \item[(3c)] Il prodotto finito di $n$ ideali è tale che 
  	  \begin{equation*}
	  	  I_1 \cdots I_n = (\left\{a_1 \cdots a_n \, \middle|\, a_1 \in I_1, \dots, a_n \in I_n\right\})
  	  \end{equation*}
      \item[(3c)] Definizione di campo o corpo commutativo.
      \item[(3c)] Definizione di dominio di integrità.
    \end{enumerate}

    \subsection{Lezione 4}

    Le lezioni sono disponibili sottoforma di 
    \href{https://didatticaonline.unitn.it/dol/course/view.php?id=23268}{videolezioni
    su moodle}.
    
    \begin{enumerate}
      \item[4d] Ripasso ideale massimale e primo.
      \item[4d] Proprietà degli ideali:
        \begin{enumerate}
          \item Dato $M \subset A$ è massimale sse $A / M$ è un campo.
          \item Sia $I \subset M$ ideale $I$ è primo sse $A / I$ è un dominio di
            integrità.
        \end{enumerate}
      \item[4d] Se $A$ è un dominio di integrità allora anche $A\left[t\right]$
        è un dominio di integrità.
      \item[4d] (Per induzione) Se $A$ è un dominio di integrità allora anche $A\left[t_1,
        \dots,t_n\right]$ è un dominio di integrità.
      \item[4e] Dimostrazione dei precedenti due enunciati.
      \item[4e] Definizione di grado di un polinomio.
      \item[4e] Enuncia alcuni esercizi svolti:
        \begin{enumerate}
          \item $K$ è un campo se e solo se ha un unico ideale proprio, ovvero
            $\left\{ 0 \right\}$.
          \item Se $A$ anello qualsiasi, allora $\left[t\right]$ non è un campo.
          \item Se $A\left[t\right]$ è un dominio allora $A$ è un dominio.
          \item $A$ è dominio se e solo se $A\llbracket t \rrbracket$ è un dominio.
        \end{enumerate}
      \item[4f] Sia $I \subseteq A$ allora si dice radicale di $I$, ovvero
        $\sqrt{I}$ definito come
            \begin{equation*}
              \sqrt{I} \coloneqq \left\{ a \in A \,\middle|\, \exists n > 0,
                a^n \in
              I\right\}
            \end{equation*}
      \item[4f] Dimostrazione di proprietà sul radicale
        \begin{enumerate}
          \item Se $I$ ideale allora $I \subseteq \sqrt{I}$.
          \item $\sqrt{I}$ è un ideale.
          \item $\sqrt{I} = A$ se e solo se $I = A$.
          \item $\sqrt{\sqrt{I}} = \sqrt{I}$.
          \item Se $I \subseteq J \subseteq A$, con $I,J$ ideali allora
            $\sqrt{I} \subseteq \sqrt{J}$.
          \item In particolare se vale una inclusione stretta, non è detto che
            valga l'inclusione stretta dei radicali. Si prenda $\mathbb{Z}$
            e gli ideali $(p^k), (p)$, sono tali che $(p^k) \subseteq (p)$ ma
            $\sqrt{(p^k)} = \sqrt{(p)} = (p)$.
        \end{enumerate}
       \item[4f] Si dice nilradicale di $A$ l'ideale $\sqrt{0}$. Gli elementi
         nel nilradicale si dicono nilpotenti.
       \item[4f-g] I domini non hanno elementi nilpotenti non banali. Alcuni
         esempi di anelli con elementi nilpotenti. 
    \end{enumerate}

    
    \subsection{Lezione 5}

    Le lezioni sono disponibili sottoforma di 
    \href{https://didatticaonline.unitn.it/dol/course/view.php?id=23268}{videolezioni
    su moodle}.
    
    \begin{enumerate}
        \item[5a] Ripasso operazioni su ideali.
        \item[5a] Dati due ideali $I,J \subseteq A$, allora vale che 
            \begin{equation*}
              IJ = \left\{ \sum^n_{i=1} a_i b_i \,\middle|\, a_i \in I, b_i \in
              I, \forall i \in \left\{0, \dots, n \right\} \forall n \in \mathbb{N}\right\}
            \end{equation*}
        \item[5a] Siano $I,J,H \subseteq A$ ideali, allora 
              \begin{equation*}
                  I(J+H)= IJ + IH
              \end{equation*}
        \item[5b] Teoremi cinesi del resto. 
        \item[5b] Due ideali $I,J$ si dicono coprimi se $I+J = A$. In
          particolare se esistono $\alpha \in I, \beta \in J$ tali che $\alpha
          \beta = 1$.
        \item[5b] Dimostrazione del teorema cinese del resto nel caso di due
          ideali. Dati due ideali $I,J$ se essi sono coprimi allora 
            \begin{align*}
              I \cap J & = IJ \\
              A / IJ & = A / (I \cap J) \simeq A/I \times A/J 
            \end{align*}
        \item[5b] Definizione del prodotto cartesiano di anelli
          e dimostrazione di buona definizione, poi continuazione della
          dimostrazione del teorema cinese dei resti.
        \item[5c] Dimostrazione nel caso di un numero arbitrario finito del
          teorema cineses del resto.
        \item[5c] `Il trucco' per dimostrare il caso con un numero arbitrario
          finito di ideali, ovvero usare $I_2 \cdots I_n$ come ideale coprimo
          con $I_1$ e poi usare induzione e il teorema cinese dei resti nel caso
          di due ideali.
        \item[5c] Definizione: $S \subseteq A$ si dice 
          moltiplicativo se $1 \in S$ e se $s, t \in S$ allora $st \in S$.
        \item[5c-d] $I$ è massimale sse è un ideale proprio e per ogni $a \in
          A \setminus I$ allora $(a, I) = A$.
    \end{enumerate}


    \subsection{Lezione 6}

    Le lezioni sono disponibili sottoforma di 
    \href{https://didatticaonline.unitn.it/dol/course/view.php?id=23268}{videolezioni
    su moodle}. L'audio sembra avere un focolare acceso in prossimità della
    videocamera.
    
    \begin{enumerate}
      \item[6e] Enunciato del Teorema di Krull-Zorn: se $A \neq \{0\}$
        anello allora ha ideali massimali contenti un ideale $I \subset A$ 
        (richiede AC via Lemma di Zorn).
      \item[6e] Definizione di ordinamento parziale e totale, catena. 
      \item[6e] Enunciato del Lemma di Zorn (inteso come assioma equivalente
        all'assioma della scelta): Sia $E$ un insieme parzialmente
        ordinato rispetto a una relazione $\subseteq$ allora se ogni catena ha
        un maggiorante, dev'essere che $E$ ha un elemento massimale. 
      \item[6e-f] Dimostrazione del teorema di Krull-Zorn. 
      \item[6f] In particolare dal Teorema di Krull-Zorn si ottiene che esistono
        sempre ideali primi, ovvero $\operatorname{Spec}(A) \neq
        \varnothing$ per ogni $A \neq \{0\}$. 
      \item[6f] Alcune proprietà degli ideali primi rispetto alle mappe
        \begin{enumerate}
          \item Sia $f \colon A \to B$ omomorfismo di anelli. Se $J$ è un
            ideale di $B$, allora $f^{-1}(J)$ è un ideale.
          \item Se $J$ è primo allora $f^{-1}(J)$ è primo.
          \item Esempio: $\mathbb{Z} \overset{i}{\subseteq} \mathbb{Q}$, ma
            $i(\mathbb{Z})$ non è un ideale. Non vale che $i^{-1}(0) \subset
            \mathbb{Z}$ è massimale malgrado $0$ lo sia in $\mathbb{Q}$.
      \end{enumerate}
    \item[6g] Proprietà degli omomorfismi quozienti  
        \begin{enumerate}
          \item Sia $I \subset A$ e $I \neq A$ e sia $\pi \colon A \to
            A / I$ la proiezione canonica. Allora c'è biezione tra gli ideali
            massimali di $A$ che contengono $I$ e gli ideali massimali di
            $A / I$.   
          \item (Nello stesso contesto del punto precedente) Esiste una biezione 
            tra ideali primi di $A$ che contengono $I$ e gli ideali primi di $A/I$.
          \item In generale le mappe surgettive si comportano bene tra $f \colon
            A \to A / \ker(f)$ (ovvero mantengono le proprietà degli ideali).
        \end{enumerate}
      \item[6g] Sia $A$ anello e $S \subseteq A$ insieme moltiplicativo 
        tale che esiste $I$ ideale di $A$ con $I \cap S \neq \varnothing$. Allora esiste 
        ideale primo $P(S)$ che contiene $I$ e $P \cap
        S = \varnothing$\footnote{richiede l'assioma della scelta}. 
      \item[6g] Ripasso del radicale e nilradicale.
      \item[6h] Enunciato: sia $I \subseteq A$ e $\pi \colon A \to A / I$ con $I 
        \subseteq J$, allora $\sqrt{J/I} = \sqrt{J}/I$.
    \end{enumerate}

    \subsection{Lezione 7}

    Le lezioni sono disponibili sottoforma di 
    \href{https://didatticaonline.unitn.it/dol/course/view.php?id=23268}{videolezioni
    su moodle}.
    
    \begin{enumerate}
      \item[7i] Definizione di ideale radicale, ovvero gli ideali $I \subseteq
        A$ tali che $\sqrt{I} = I$.
      \item[7i] Gli ideali primi sono radicali.
      \item[7i] Dimostrazione della proposizione: $I \subseteq A$ ideale, allora 
        \begin{equation*}
          \sqrt{I} = \bigcap_{\substack{P \in \operatorname{Spec}(A),\\ I \subseteq P}} P
        \end{equation*}
      \item[7i] L'intersezione di radicali è radicale. 
      \item[7i] Ogni ideale proprio $I \neq A$ è contenuto in un ideale
        massimale sse esiste un ideale massimale. Questo è ovvio considerando il 
        quoziente $A/I$ e usando il lemma di Krull-Zorn (debole).
      \item[7i] $A$ si dice \emph{locale} se ha un unico ideale massimale
        (più esempi nel capitolo 4 di \cite{vergura})
      \item[7i-j] Esempio delle serie formali come anelli locali
        e caratterizzazione dei suoi ideali. 
      \item[7j] Dimostrazione che ogni anello di polinomi su un campo è un dominio 
        e un anello a ideali principali (si può vedere in modo analogo a quanto 
        fatto in Algebra A con i domini euclidei su $\mathbb{Z}$). 
      \item[7k] Ideali primi su $\mathbb{R}\left[ t \right]$. 
    \end{enumerate}
    
    \section{Moduli}
    
    \subsection{Lezione 7}

    Le lezioni sono disponibili sottoforma di 
    \href{https://didatticaonline.unitn.it/dol/course/view.php?id=23268}{videolezioni
    su moodle}.
    
    \begin{enumerate}
      \item[7k] Introduzione alla teoria dei moduli come ``spazi vettoriali per
        gli anelli'': fissato $A$ anello e $M$ si dice $A$-modulo si dice modulo 
        se $(M,+,0_M)$ è un gruppo abeliano e esiste l'operazione $\cdot \colon 
        A \times M \to M$. L'operazione $\cdot$ dev'essesere distributiva,
        pseudo-associativa, avere un elemento neutro.
      \item[7k] Se $A = \{0\}$ allora l'unico $A$-modulo è $\{0\}$. 
      \item[7k] Se $M$ è un $A$-modulo, allora $0 \in N \subseteq M$ si dice
        sottomodulo se è chiuso rispetto al $+$ del modulo e chiuso rispetto al
        prodotto esterno dall'anello $A$.
      \item[7k] Esempi di moduli: fissato un anello questo è un modulo con
        sottomoduli gli ideali dell'anello.
    \end{enumerate}


    \subsection{Lezione 8}

    Le lezioni sono disponibili sottoforma di 
    \href{https://didatticaonline.unitn.it/dol/course/view.php?id=23268}{videolezioni
    su moodle}.
    
    \begin{enumerate}
      \item[8a] Ripasso della definizione di $A$-modulo, sottomodulo e osservazioni 
        banali.
      \item[8a] L'intersezione arbitraria di sottomoduli è un sottomodulo.
      \item[8a] Sia $S \subseteq M$ con $M$ $A$-modulo. Allora il
        sotto-$A$-modulo generato da $S$ è 
            \begin{equation*}
              (S) \coloneqq \bigcap_{S \subset N \subseteq M} N
            \end{equation*}
        con $N$ sottomoduli. 
      \item[8a] $(S)$ è il minimo modulo di $N$ tale che contiene $S$.
      \item[8a] Sia $M$ un $A$-modulo e $N_1, N_2 \subseteq M$ sottomoduli.
        Allora definiamo 
        \begin{equation*}
          N_1 + N_2 = (N_1 \cup N_2)
        \end{equation*}
      \item[8b] Siano $M, N$ $A$-moduli. Allora $f \colon M \to N$ si dice
        omomorfismo di $A$-moduli (anche detta applicazione $A$-lineare) 
        se rispetta le operazioni:
        \begin{enumerate}
          \item È un omomorfismo di gruppi abeliani, vale 
                $f(a+b) = f(a)+f(b) = f(b)+f(a) = f(b+a)$
          \item Preserva il prodotto dell'anello sul modulo, vale
                $f(\lambda n) = \lambda f(n)$
        \end{enumerate}
      \item[8b] Alcune proprietà degli omomorfismi tra $A$-moduli:
        \begin{enumerate}
          \item La controimmagine di un sottomodulo attraverso un omomorfismo di
            moduli è un sottomodulo.
          \item L'immagine di un omomorfismo di moduli è un sottomodulo del
            codominio.
          \item $\ker(f)$ è un sottomodulo.
        \end{enumerate}           
      \item[8b] Definizione del quoziente su $A$-moduli (totalmente analogo ai
        gruppi), buona definizione delle operazioni di $A$-modulo su quoziente
        e della mappa di proiezione canonica.
      \item[8c] Alcuni esercizi (analoghi a quelli per ideali/anelli)
        \begin{enumerate}
            \item Sia $f \colon M \to N$ omomorfismo bigettivo allora
                è isomorfismo di $A$-moduli.
            \item Sia $L \subseteq M$ sotto-$A$-modulo, allora $\pi : M \to M/L$ da
                una bigezione tra i sotto-$A$-moduli contenenti $L$ e i 
                sotto-$A$-moduli di $M/L$.
            \item Dati $L_1,L_2,L_3 \subset M$, $A$-sottomoduli allora
              \begin{equation*}
                (L_1 \cap L_2) + (L_1 \cap L_3) \subseteq L_1 \cap (L_2 + L_3) 
              \end{equation*}
            \item Siano $L_1 \subseteq L_2 \subseteq M$ sotto-$A$-moduli, allora
              \begin{equation*}
                (M/L_1)/(L_2/L_1) \simeq M/L_2
              \end{equation*}
      \end{enumerate}
      \item[8c] Definizioni di operazioni tra due $A$-moduli $M,N$:
        \begin{enumerate}
          \item Definiamo 
            \begin{equation*}
              \operatorname{Hom}_A(M,N) \coloneqq \left\{ f \colon M \to
              N \,\middle|\, f\, \text{omomorfismo tra moduli} \right\}
          \end{equation*}
          allora è un $A$-modulo.
        \item Definizione di 
          \begin{equation*}
            \operatorname{End}_A(M) \coloneqq \operatorname{Hom}_A(M,M)
          \end{equation*}
          che è ancora un $A$-modulo. Inoltre ha la composizione come operazione
          associativa, distributiva e forma un anello non-commutativo rispetto
          a $(+, \circ)$.
        \end{enumerate}
      \item[8d] Mappa identità come unità dell'anello $\operatorname{End}(M)$.
    \end{enumerate}

    \subsection{Lezione 9}

    Le lezioni sono disponibili sottoforma di 
    \href{https://didatticaonline.unitn.it/dol/course/view.php?id=23268}{videolezioni
    su moodle}.
    
    \begin{enumerate}
      \item[9e] Proposizione 2.1.3\cite{vergura}: Siano $A$ un anello e $M$ 
        un gruppo abeliano. Allora ogni struttura di $A$-modulo su $M$ 
        definisce un omomorfismo di anelli $\mu \colon A \to 
        \operatorname{End}(M)$ e viceversa.
      \item[9e] Il nucleo della mappa della proposizione 2.1.3 è un ideale di
        $A$, quindi $A/\ker(\mu)$. (Esercizio: quando $a \in \ker(\mu)$ se
          e solo se $am = 0$ per ogni $m \in M$, viene anche detto
        $\operatorname{Ann}(M)$ annichilatore o annullatore)
      \item[9e] Osservazione è possibile costruire una struttura di modulo anche
        su $A/\ker(\mu)$.
      \item[9e] Costruzioni di moduli
        \begin{enumerate}
          \item Presa una qualunque famiglia non vuota $\left\{ M_\alpha
            \right\}_{\alpha \in \Gamma}$ di $A$-moduli, allora il prodotto 
            cartesiano di $A$-moduli è definito come 
            \begin{equation*}
              \prod_{\alpha \in \Gamma} M_\alpha = \left\{ (m_1, \dots,)
              \,\middle|\, m_\alpha \in M_\alpha \right\}
            \end{equation*}
            si possono definire somma e prodotto rispetto all'anello $A$. Da cui
            si può determinare una struttura di $A$-modulo. Definizione delle
            proiezioni come omomorfismi tra l'$A$-modulo prodotto a una delle
            componenti.
          \item Presa una qualunque famiglia non vuota $\left\{ M_\alpha
            \right\}_{\alpha \in \Gamma}$ di $A$-moduli, allora definiamo la
            somma diretta come 
            \begin{equation*}
              \bigoplus_{\alpha \in \Gamma} M_\alpha \coloneqq \left\{
                (m_\alpha)_{\alpha \in \Gamma} \,\middle|\, \text{un numero
                finito di termini}\, m_\alpha \neq 0 \right\} \subseteq
                \prod_{\alpha \in \Gamma} M_\alpha
            \end{equation*}
          con le solite operazioni di somma e prodotto (come nel prodotto
          cartesiano). Allora la somma diretta è un sottomodulo del prodotto
          cartesiano.
        \end{enumerate}
      \item[9f] La somma diretta di $\left\{ M_\alpha \right\}_{\alpha \in
        \Gamma}$ è il minimo sottomodulo del prodotto cartesiano, generato da
        \begin{equation*}
          \bigcup_{\alpha \in \Gamma} \tilde{M}_\alpha \quad \tilde{M}_\alpha
          \coloneqq i_\alpha(M_\alpha) 
        \end{equation*}
        dove $i_\alpha \colon M_\alpha \to \prod_{\alpha \in \Gamma} M_\alpha$
        è l'iniezione canonica nel prodotto cartesiano. 
      \item[9f-g] Dimostrazione che il modulo $\operatorname{Hom}_A(A^n, A^m)$ con
        $A$ anello e $m,n \in \mathbb{N}$. Allora gli elementi del modulo
        $\operatorname{Hom}_A(A^n, A^m)$ si rappresentano come matrici $m \times
        n$ (i.e. c'è un omomorfismo tra moduli tra il modulo delle matrici
        e quello degli omomorfismi tra prodotto di anelli). 
      \item[9g] Dato $M$ $A$-modulo, si dice libero se esiste una base
        $(m_\alpha)_{\alpha \in \Lambda}^n$. Ovvero se la base genera $M$ 
        (attraverso somme finite) e vale la condizione di lineare indipendenza 
        (di ogni suo sottoinsieme finito):
        \begin{equation*}
          \sum_{\alpha = 0}^n a_\alpha m_\alpha = 0 \implies \forall \alpha . \, a_{\alpha} = 0 
        \end{equation*}
      \item[9g] (Esercizio) Se $M \neq \left\{ 0 \right\}$ è un $A$-modulo, $M$
        ha base finita (di dimensione $n$) se e solo se $A^n \cong M$. 
      \item[9g] Esempio di modulo senza base: $A = \mathbb{Z}$ e $M
        = \mathbb{Z}/2\mathbb{Z}$; senza base finita $A = \mathbb{Z}$ e 
        $M = \mathbb{Q}$.
      \item[9g] Se $M$ ha base del tipo $\left\{ l_\alpha \right\}_{\alpha \in
        \Omega}$ in bigezione con $\Gamma \neq \varnothing$ se
        e solo se $M \cong A^{(\Gamma)} = \bigoplus_{\alpha \in \Gamma} A$ 
      \item[9g-h] Esempi 2.2.1\cite{vergura}. 
    \end{enumerate}

    \subsection{Lezione 10}

    Le lezioni sono disponibili sottoforma di 
    \href{https://didatticaonline.unitn.it/dol/course/view.php?id=23268}{videolezioni
    su moodle}.
    
    \begin{enumerate}
      \item[10i] Continuazione Esempi 2.2.1\cite{vergura}. 
        \begin{enumerate}
          \item Dimostrazione che dati tre $A$-moduli, $M_1, M_2, N$, allora 
            \begin{equation*}
                \operatorname{Hom}_A(M_1 \oplus M_2, N) \cong
                \operatorname{Hom}_A(M_1, N) \oplus  \operatorname{Hom}_A(M_2, N)
            \end{equation*}
          \item Dimostrazione che dati tre $A$-moduli, $M, N_1, N_2$, allora 
            \begin{equation*}
                \operatorname{Hom}_A(M, N_1 \oplus N_2) \cong
                \operatorname{Hom}_A(M, N_1) \oplus  \operatorname{Hom}_A(M, N_2)
            \end{equation*}
        \end{enumerate}
      \item[10i] Dati $M,N,L$ $A$-moduli. Allora $M \to N \to L$ è un complesso
        se $g \circ f = 0$, ovvero $f(M) \subseteq \ker(g)$ (in generale possono
        essere anche un infinità numerabile di $A$-moduli e mappe).  
      \item[10i] Dato un complesso, la successione si dice esatta se $f_i(M)
        = \ker(f_{i+1})$.
      \item[10i] Le successioni esatte interessanti sono quelle del tipo 
        \begin{equation}
          \label{eq:succ_esatta_brv}
          0 \to M \xrightarrow{\alpha} N \xrightarrow{\beta} L \to 0
        \end{equation}
        se è esatta si dice successioni esatte brevi.
      \item[10j] Osservazioni sulle successioni esatte brevi (si veda Esempi
        2.3.1 \cite{vergura}): 
        \begin{enumerate}
          \item Una successione breve del tipo 
            \begin{equation*}
              0 \to L \xrightarrow{\alpha} M
            \end{equation*}
            è esatta se e solo se $\ker{\alpha} = \left\{ 0 \right\}$.
          \item Una successione breve del tipo 
            \begin{equation*}
              M \xrightarrow{\beta} N \xrightarrow 0
            \end{equation*}
            è esatta se e solo se $\beta$ è suriettiva.
          \item Inoltre se si ha una sequenza del tipo
            \eqref{eq:succ_esatta_brv} allora è esatta se e solo se $\alpha$
            iniettiva, $\beta$ suriettiva e $\ker(\beta) = \alpha(L)$.
        \end{enumerate}
      \item[10j] Dati due moduli $L, N$, allora possiamo fare una successione
        esatta \footnote{è l'equivalente del teorema nullità+rango}
        \begin{equation*}
          0 \to L \to L \oplus N \to N \to 0
        \end{equation*}
      \item[10j] Una successione esatta si dice che splitta (spezza) se data una
        successione esatta del tipo \eqref{eq:succ_esatta_brv} allora $N \cong
        M \oplus L$.
      \item[10j] (*) Se esiste un omomorfismo di moduli $\varphi \colon M \to L \oplus N$ 
        tale da far commutare il diagramma 
        \begin{equation*}
        \begin{tikzcd}
        	0 \arrow{r} & L \arrow{r}{\alpha} \arrow{d}{Id_L} & M \arrow{r}{\beta}
        	\arrow[dashed]{d}{\varphi} & N \arrow{r} \arrow{d}{Id_N} & 0 \\
        	0 \arrow{r} & L \arrow{r}{\alpha} & L \oplus N \arrow{r}{\beta} 
        	& N \arrow{r} & 0 \\
        \end{tikzcd}	
        \end{equation*}
        allora $\varphi$ è un isomorfismo.
      \item[10k] Se $N$ è un $A$-modulo libero allora ogni successione corta
        esatta con $N$ all'ultimo passo (prima dello $0$), è una successione
        spezzante. Dimostrazione nel caso generale in cui $N$ non ha base
        finita. (Dimostrazione non conclusa) 
      \item[10k] Se $A$ è un campo e si abbiano spazi vettoriali di dimensione
        finita $V_1, V_2, V_3$, allora se la successione è esatta 
        \begin{equation*}
          0 \to V_1 \to  V_2 \to V_3 \to 0
        \end{equation*}
        allora $\dim(V_2) = \dim(V_1) + \dim(V_3)$. (senza dimostrazione, ma
        semplice corollario del teorema precedente)
      \item[10k] Se $A$ è un campo e si hanno $\{V_i\}_{i = 0}^n$ 
     	spazi vettoriali che formano una successione esatta 
        \begin{equation*}
          0 \to V_1 \to \cdots \to V_n \to 0
        \end{equation*}
        allora $\sum_{i = 0}^n (-1)^i \dim(V_i) = 0$. Con dimostrazione. 
      \item[10l] Continuazione dimostrazione della proposizione precedente.
    \end{enumerate}   
    
    \section{Noethrianità}
    
    \subsection{Lezione 11 - parte 1}

	Le lezioni sono disponibili sottoforma di 
	\href{https://didatticaonline.unitn.it/dol/course/view.php?id=23268}{videolezioni
		su moodle}.
	
	\begin{enumerate}
      \item[11a] Proposizione 3.1.1\cite{vergura}: $A$ anello, allora le seguenti affermazioni
        sono equivalenti (assumendo l'assioma della scelta):
        \begin{enumerate}
          \item Ogni famiglia non vuota di ideali di $A$ ha un elemento
            massimale\footnote{Non dice che esiste un ideale massimale di $A$
              nella famiglia di ideali. Il massimale è inteso solo rispetto alla
            relazione di $\subset$}.
          \item Ogni catena ascendente di ideali $\left\{ I_k
          \right\}_{k\in \Gamma}$ di $A$  è stazionaria ovvero
            \begin{equation*}
              I_1 \subsetneq I_2 \subsetneq \dots \subsetneq I_{n+1}
              = I_{n+1} = \dots 
            \end{equation*}
            per un certo $n > 1$.
          \item Ogni ideale $I$ di $A$ è finitamente generato.
        \end{enumerate}
      \item[11a] Sia $A$ anello si dice noetheriano se soddisfa una delle
        condizioni precedenti.
      \item[11a-b] Dimostrazione della Proposizione 3.1.1\cite{vergura}.
      \item[11b] Esercizio: Sia $A$ noetheriano $I \subset A$
        ideale di $A$, allora $A/I$ è noetheriano.
      \item[11c] Esempio di anello non noetheriano (Esempio 3.1.1)\cite{vergura}.
      \item[11c] Esempio 3.2.1\cite{vergura}: anello di polinomi $\mathbb{K}\left[ x,y \right]$,
        dimostra che fissato un intero $k$, allora si può sempre trovare un
        ideale che richiede $k+1$ elementi.
      \item[11c] Storia dell'algebra commutativa, teorema della base di Hilbert
        e sviluppi successivi da parte di Noether e Krull.
	\end{enumerate}   

    \subsection{Lezione 11 - parte 2}

	Le lezioni sono disponibili sottoforma di 
	\href{https://didatticaonline.unitn.it/dol/course/view.php?id=23268}{videolezioni
		su moodle}.
	
	\begin{enumerate}
      \item[11d] Teorema 3.2.1: Teorema della base di Hilbert: $A$ anello
        noetheriano se e solo se $A\left[t\right]$ è noetheriano (con
        dimostrazione; si legga 
        \url{https://en.wikipedia.org/wiki/Hilbert\%27s_basis_theorem}). 
      \item[11e] Corollario 3.2.1 
      \item[11e] Alcuni esempi di anelli noetheriani: 
        \begin{enumerate}
          \item $\mathbb{Z}$, $\mathbb{Z}\left[x_1, \dots,x_n\right]$.
          \item $A_1, A_2$ anelli noetheriani allora $A_1 \times A_2$ sono
            noetheriani. 
        \end{enumerate}
      \item[11e] Enunciato / Esercizio: Se $A$ noetheriano allora $A\llbracket x \rrbracket$ 
        è noetheriano.
      \item[11e] Definizione 3.2.1: Sia $A$ anello, si dice $A$-algebra una
        coppia ordinata $(B,\varphi)$ con $B$ anello e $\varphi \colon A \to B$
        omomorfismo. (spiegazione della definizione: praticamente $B$ diventa un
        $A$-modulo con $B$ anello rispetto al prodotto $\varphi(a) b$ con $a \in
        A, b\in B$). Se $\varphi$ è iniettiva, allora è anche detta estensione di
        anelli.
      \item[11e] Definizione 3.2.4: Una $A$-algebra $(B, \varphi)$ si dice
        finita se è finita come $A$-modulo. Si dice invece finitamente generata
        se $\exists b_1, \dots, b_k \in A$ tali che l'omomorfismo di $A$-moduli
        \begin{equation*}
        	\begin{tikzcd}[row sep=0.1em, column sep = 2em]
        		f \colon   & A\left[ x_1, \dots,x_k \right] \arrow{r} & B \\
        	 			& (a_n x_i^n + \dots + a_0) \arrow[r, maps to] 
        	 			& \varphi(a_n)b^n_i + \dots \varphi(a_0)		
        	\end{tikzcd}
        \end{equation*}
        sia suriettivo. In tal caso scriviamo $B = A\left[ b_1, \dots, b_k
        \right]$.
      \item[11e] Proposizione 3.2.1: Sia $A$ anello noetheriano e sia
        $(B,\varphi)$ una $A$-algebra. Allora
        \begin{enumerate}
          \item Se $B$ è finita, allora $B$ è un $A$-modulo noetheriano.
          \item Se $B$ è finitamente generata, allora $B$ è un anello
            noetheriano.
        \end{enumerate}
	\end{enumerate}
   
    \subsection{Lezione 12 - parte 1}

	Le lezioni sono disponibili sottoforma di 
	\href{https://didatticaonline.unitn.it/dol/course/view.php?id=23268}{videolezioni
		su moodle}.
	
	\begin{enumerate}
		\item 
	\end{enumerate}   

	\subsection{Lezione 12 - parte 2}
	
	Le lezioni sono disponibili sottoforma di 
	\href{https://didatticaonline.unitn.it/dol/course/view.php?id=23268}{videolezioni
		su moodle}.
	
	\begin{enumerate}
		\item 
	\end{enumerate}  

	\subsection{Lezione 13 - parte 1}
	
	Le lezioni sono disponibili sottoforma di 
	\href{https://didatticaonline.unitn.it/dol/course/view.php?id=23268}{videolezioni
		su moodle}.
	
	\begin{enumerate}
		\item 
	\end{enumerate}  

	\subsection{Lezione 13 - parte 2}
	
	Le lezioni sono disponibili sottoforma di 
	\href{https://didatticaonline.unitn.it/dol/course/view.php?id=23268}{videolezioni
		su moodle}.
	
	\begin{enumerate}
		\item 
	\end{enumerate}  

    \begin{thebibliography}{9}
      \bibitem{vergura} M. Vergura, \textit{Dispense di Algebra commutativa},
        \url{www.science.unitn.it/~ballico/Algebra_Commutativa-Ballico14.pdf}.
      \bibitem{milne} J.S. Milne, \textit{A Primer of Commutative Algebra}
      \bibitem{atyiah-macdonald} M.F. Atiyah, I.G. Macdonald, \textit{Introduction to 
            Commutative Algebra}.
      \bibitem{reid} M. Reid, \textit{Undergraduate Commutative Algebra}.
    \end{thebibliography}
\end{document}


